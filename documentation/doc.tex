\documentclass[12pt,a4paper]{article}
\usepackage[utf8]{inputenc}
\usepackage{amsmath}
\usepackage{amsfonts}
\usepackage{amssymb}
\usepackage[T2A]{fontenc}
\usepackage{tipa,tipx}
\inputencoding{utf8}
\usepackage{textgreek}
\usepackage{graphicx}
\usepackage[printonlyused]{acronym}
\usepackage{hyperref}

\author{wediaklup}
\title{Caseyan Dictionary v1.0}
\date{27 March 2022}

\begin{document}

\maketitle
\tableofcontents

\section{Introduction}
This software was developed by me, \textbf{wediaklup a.k.a. Diamant}, and published to GitHub in order to simplify working with Caseyan text. It must not be published to other sites without my permission. \\
The software is divided into multiple programs and files: \textit{decl.py} — Internal Method —, \textit{dicteng.py} — Handles words in \textit{liste.kaz} —, \textit{dictionary.py} — The part where the user can look for words — and \textit{liste.kaz} — the word list. While there are a few more files, they are not relevant to the user.

\section{Functions}
\subsection{Word Search}
test

\end{document}