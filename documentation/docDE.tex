\documentclass[12pt,a4paper]{article}
\usepackage[ngerman]{babel}
\usepackage[utf8]{inputenc}
\usepackage{amsmath}
\usepackage{amsfonts}
\usepackage{amssymb}
\usepackage[T2A]{fontenc}
\usepackage{tipa,tipx}
\inputencoding{utf8}
\usepackage{textgreek}
\usepackage{graphicx}
\usepackage[printonlyused]{acronym}
\usepackage{hyperref}

\author{wediaklup}
\title{Weiß der Geier wie das heißen wird}

\begin{document}

\maketitle
\tableofcontents

\section{Einleitung}
Diese Software wurde von mir, \textbf{Diamant}, erstellt und auf GitHub hochgeladen, um das Arbeiten mit der kaseiischen Sprache zu vereinfachen. Es darf ohne meine Erlaubnis nicht auf anderen Seiten hochgeladen werden. \\
Im wesentlichen besteht das Produkt aus drei Programmen: dem Deklinator – \textit{decl.py}, der Engine – \textit{dicteng.py} und dem eigentlichen Wörterbuch – \textit{dictionary.py}. In Zukunft sollen auch ein Synonymwörterbuch und ein Textübersetzer enstehen.

\end{document}